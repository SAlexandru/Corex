\chapter{Conclusions}\label{conclusions}

\section{Summary}

        As mentioned from the begining, our goal when building XCore was to create a tool / framework
for easily building extensible analysis tools without sacrificing any of the type safety properties
of the language. 

        Most analysis tools are build by creating a very complex meta-model, which has to strive to be
extensible, in order to adapt easily to changes that might come in the future. Unfortunatelly this task
of building a meta-model, is problematic, not only for its sheer complexity, but also for the fact that 
it makes it very hard, if not impossible to integrated different analysis tools together. 

        One tool, that we presented in the introduction and in \cite{oldThesis} is CodePro. It has tried
building a complex meta-model and to ensure extensibility it has sacrificed type safety. One consequence is
that there was no way to staticly ensure what type of entity can be analysed by a given metric. Everyone received an entity
as type \code{Object}. Basically it has transformed a type safety language into a dynamic language. 

        Our solution is to automate the process of creation of the meta-model. We have developed a meta-meta-model
which allows the tool developers to describe their own meta-model. The meta-model can be described using java annotations, as presented in
chapter \ref{ch:2}. For now our meta-meta-model supports  metrics (we call them properties), inclusion relationships and actions.
Also we do not provide any means of directly extracting the described entities from the projects source code. In order to solve this issue
we allow users to integrate with any back-end they want.  Also we provide a way of integrating multiple back-ends. We have a primary, concrete
entity type that can be set and any other secondary such types can be selected, but a transformer from the primary to the secondary type has
to also be provided. One important aspect, all of this happens at compile time ! All the provided annotations are expanded at compile making
XCore type safe which we also enforce on the analysis tools. 

        For the porpouse of the evaluation we have previously created a tool caled \code{XCoreView}. It's basically an Eclise Widget which 
every XCore tool can hook up an use. It is based on the work done on \code{InsiderView} from CodePro \cite{tools:inCode}.

\section{Future Work}

        We are currently working on providing a sandbox system for adding / integrating other tools into the our main / core tool that we have build. 
This provides a nice way for a team of developers to work togheter and integrate with other without too many problems. Also it will provide the ability
to isolate componentst that do not work together. 
